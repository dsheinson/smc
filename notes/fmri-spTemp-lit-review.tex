\documentclass{article}

\usepackage{natbib}

\title{Spatio-temporal modeling of fMRI - overview of Previous Work}

\begin{document}

\maketitle

\noindent {\bf \citet{zhang:guin:2014:npBayesWave}} \\
\noindent Presents a wavelet-based Bayesian nonparameteric spatio-temporal model for fMRI data. Provides a joint analytical framework for detecting regions of neuronal activity as well as simultaneously inferring the clustering of spatially remote voxels that exhibit fMRI time series with similar characteristics. Models the HRF with a voxel-depenendent shape paramter that determines the delay in response. Uses mixture priors with a spike at zero on regression coefficients and induces spatial correlation using a Markov random field. Use discrete wavelet transforms to assume long-memory correlation in time courses and achieve clustering of voxels by imposing a Dirichlet process prior on the parameters of the long-memory process. Use M-H Bayesian variable selection sampling techniques combined with sampling algorithms for nonparametric DP models for inference on simulated block and event-related design data, and real fMRI data. \\

\noindent {\bf \citet{kang:wang:2011:vpr}} \\
\noindent Uses variable parameter regression and Kalman filtering to construct a dynamic model for connectivity in the brain at resting state. Uses a temporally nonstationary random walk model for the regression coefficient that represents the correlation between two voxel time series. Identifies dynamic resting-state functional connectivity (RSFC) systems and examines the dynamic pattern of activation within networks and between networks. Shows that 10-20\% of all voxels within each RSFC brain map show significant activity throughout the experiment. Shows that the spatial pattern of dynamic connectivity is similar within and between maps. \\

\noindent {\bf \citet{quiros:diez:2010:BayesSpTemp}} \\
\noindent Introduces a Bayesian spatio-temporal model for data collected from a block-design fMRI experiment. Parameterizes a voxel-specific hrf by a Poisson distribution function with parameter interpreted as the delay in the impulse response before an increase in signal. Uses a Gaussian random field prior on the location of activations and a Gaussian CAR prior on the magnitude of responses at each location. Uses Gibbs sampling and Metropolis-Hastings steps to analyze simulated, synthetic, and real data via MCMC.

\noindent {\bf \citet{bowman:2008:SpMCMC}} \\
\noindent Develops a spatial Bayesian hierarchical model for making inferences regarding task-related changes in brain activity and identifies task-related connectivity using MCMC techniques via Gibbs sampling. Studies inter-regional (long-range) correlations using an unstructured variance/covariance matrix on regional mean parameters and captures intra-regional (short-range) correlations using an exchangeable correlation structure. Uses a two-stage modeling approach to first model individual voxels via regression (using SPM2 method to deal with temporal autocorrelations) and second models regression coefficients as random effects with short-range and long-range spatial correlation components. \\

\noindent {\bf \citet{ho:ombao:2005:statespace}} \\
\noindent Formulate a state space model for simultaneously modeling brain activation and dynamic connectivity between regions of interest. fMRI data in ROIs are modeled as a linear function of the convolution of the hrf with experimental stimulus, with regression coefficients assumed to vary over time as a function of themselves and coefficients for voxels from other regions of interest at previous time points. Estimation is performed using Kalman filtering/smoothing and the EM algorithm.

\clearpage

\bibliographystyle{plainnat}
\bibliography{danny}

\end{document} 