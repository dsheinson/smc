\documentclass{article}

\usepackage{amsmath}
\usepackage{natbib}

\begin{document}

Presented here is the particle filter known as the \emph{resample-move} algorithm, as described by \citet{Gilk:Berz:foll:2001} and \citet{Berz:gilk:2001}:

Let $\theta_{j,t}$ be a vector of unknown states and/or parameters for particle $j$ at time $t$ and $w_{j,t}$ be the weight of particle $j$ at time $t$. We require that $\sum_{j=1}^J w_{j,t} = 1$, where $J$ is the total number of particles, and let $y_{1:t} = (y_1,y_2,\ldots,y_t)'$ represent the vector of data observed up until time $t$.

We allow $\theta_t$ to change with time, and we consider in particular the case where the dimension of $\theta_t$ changes with time, e.g. if $\theta_t$ contains an evolving trajectory of states. Let $\theta^{+}_t$ be the (possibly empty) set of unknown states entering the model at time $t$. Then, we have $\theta_{t+1} = (\theta_t,\theta^{+}_t)'$.

We initialize the algorithm by drawing $\theta_{j,0}$ from some prior distribution $p(\theta_0)$ and setting $w_{j,0} = 1 / J$ for $j = 1,\ldots,J$. Given a particle sample approximation of the posterior $p(\theta_{t-1}|y_{1:t-1})$ - i.e. $\left(\theta_{j,t-1}, w_{j,t-1}\right)$ for $j=1,\ldots,J$ - we move to a particle sample approximation of $p(\theta_t|y_{1:t})$ by the following steps:

\begin{enumerate}
\item Augmentation and update weights: For each $j = 1,\ldots,J$, draw $\theta^{+}_{j,t-1}$ from some state evolution distribution $p(\theta^{+}_{t-1}|\theta_{j,t-1})$ and define a new particle $\theta^{*}_{j,t-1} = (\theta_{j,t-1},\theta^{+}_{j,t-1})'$. Then, calculate the unnormalized weight $w^{**}_{j,t-1} = w_{j,t-1}p(y_t|\theta^{*}_{j,t-1})$.
\item Renormalize the $w^{**}_{j,t-1}$'s by setting $w^{*}_{j,t-1} = w^{**}_{j,t-1} / \sum_{l=1}^J w^{**}_{l,t-1}$.
\item Resample (optional): For each $j$, sample an index $k_j$ from $\{1,2,\ldots,J\}$ with associated probabilities $\{w^{*}_{1,t-1},w^{*}_{2,t-1},\ldots,w^{*}_{j,t-1},\ldots,w^{*}_{J,t-1}\}$.
\item Update weights: If resampling was performed, set $w_{j,t} = 1 / J$ for all $j$. Otherwise, set $w_{j,t} = w^{*}_{j,t-1}$.
\item Move particles (only do if `Resample' step carried out): For each $j = 1,2,\ldots,J$, draw $\theta_{j,t}$ from some transition kernel $q(\theta_t|\theta^{*}_{k_j,t-1})$ with invariant distribution $p(\theta_t|y_{1:t})$.
\end{enumerate}

\clearpage

\bibliographystyle{plainnat}
\bibliography{jarad}

\end{document} 