\documentclass{article}

\usepackage{amsmath}
\usepackage{natbib}
\usepackage{enumitem}
\usepackage{graphicx}

\begin{document}

\section{Model} \label{sec:model}

We can represent a DLM as a state space model with observation and state equations given by

\begin{align}
y_t &= Fx_t + v_t \label{eqn:obs} \\
x_t &= Gx_{t-1} + w_t \label{eqn:state}
\end{align}

\noindent where $v_t$ and $w_t$ are iid Gaussian random variables with mean 0 and covariance matrices $V$ ($q \times q$) and $W$ ($p \times p$), respectively. We also assume $v_t \perp w_{t'}$ for all $t$ and $t'$. $y_t$ ($q \times 1$) is the observed data at time $t$, $x_t$ ($p \times 1$) is the unknown state at time $t$, and $F$ ($q \times p$) and $G$ ($p \times p$) are matrices that could contain unknown parameters. Lastly, we define the distribution of the prior state $x_0$ by $x_0 \sim \mbox{N}(m_0, C_0)$, where $\mbox{N}(\mu, \Sigma)$ represents the normal distribution with mean $\mu$ and covariance $\Sigma$.

We can allow additional covariates $U_t$ into the observation equation \eqref{eqn:obs} by including a term $U_t \beta$, where $U_t$ is a $q$ by $d$ matrix and $\beta$ is a $d$ by $1$ vector. In addition, we can allow $F$ to change over time. Thus, our DLM is now written as the following:

\begin{align}
y_t &= U_t\beta + F_tx_t + v_t \label{eqn:obs2} \\
x_t &= Gx_{t-1} + w_t \label{eqn:state2}
\end{align}

\noindent Letting $\beta = 0$, $U_t = 0$ for all $t$, and $F_t = F$ independent of $t$, we have the DLM described by equations \eqref{eqn:obs} and \eqref{eqn:state}.

\subsection{Regression with AR(p) errors}

We will now consider a regression model with autoregressive error structure and separate observation and state noises. This specific model is a special case of the the DLM described in equations \eqref{eqn:obs2} and \eqref{eqn:state2}. In this case, $G$ has the form given by

\[G = \left(
\begin{array}{ccccc}
\phi_1 & 1 & 0 & \cdots & 0 \\
\phi_2 & 0 & 1 & \cdots & 0 \\
\vdots & \vdots & & \ddots & \vdots \\
\phi_{p-1} & 0 & \multicolumn{2}{c}{\cdots} & 1 \\
\phi_p & 0 & \multicolumn{2}{c}{\cdots} & 0
\end{array}
\right)\]

\noindent In addition, we let $W = \sigma^2_see'$ and $V = \sigma^2_m\tilde{V}$, where $e$ is a $p$-length vector with first element equal to 1 and the rest 0, $\sigma^2_s$ is the state or process noise, and $\sigma^2_m$ is measurement noise. We assume that the prior state $x_0$ is drawn from the stationary distribution given by $x_0 \sim \mbox{N}(0, \sigma^2_s\tilde{C_0})$, where \[\mbox{vec}({\tilde{C_0}}) = (I - G \otimes G)^{-1}\mbox{vec}(ee')\]

\section{Estimation}

Assume $\tilde{V}$, $U_t$, and $F_t$ known for all $t$. Let $\phi = (\phi_1,\ldots,\phi_p)'$, and let $\theta = (\beta', \phi', \sigma^2_m, \sigma^2_s)'$ be unknown. We now consider estimation of the joint posterior distribution of the states and unknown parameters using MCMC. Define $y_{1:t} = (y_1,\ldots,y_t)'$ to be the data observed up until time $t$. We can express the joint likelihood $p(y_{1:t},x_{0:t},\theta)$ by the following general form:

\begin{equation}
p(y_{1:t},x_{0:t},\theta) = \prod_{k=1}^t\left[p(y_k|x_k,\theta)p(x_k|x_{k-1},\theta)\right]p(x_0,\theta) \label{eqn:lik}
\end{equation}

\noindent We use independent priors on the components of $\theta$, i.e. $p(\theta) = p(\beta)p(\phi)p(\sigma^2_m)p(\sigma^2_s)$, where

\begin{align*}
&\beta \sim \mbox{N}(b_0, B_0) \qquad \sigma^2_m \sim \mbox{IG}(a_{m_0}, b_{m_0})\\
&\phi \sim \mbox{N}_{S_\phi}(\phi_0, \Phi_0) \qquad \sigma^2_s \sim \mbox{IG}(a_{s_0}, b_{s_0})
\end{align*}

\noindent The hyperparameters $b_0$, $B_0$, $\phi_0$, $\Phi_0$, $a_{m_0}$, $b_{m_0}$, $a_{s_0}$, and $b_{s_0}$ are all assumed known. $\mbox{IG}(a,b)$ represents an inverse-gamma distribution with shape parameter $a$ and rate parameter $b$. $\mbox{N}_{\Omega}(\mu, \Sigma)$ represents the normal distribution truncated onto the set $\Omega$, with mean and covariance of its corresponding untruncated normal distribution given by $\mu$ and $\Sigma$, respectively. $S_{\phi}$ represents the set of all $\phi$ such that $x_t$ comes from a stationary distribution for all $t$.

\subsection{Full conditional distributions}

We derive the following full conditional distributions for the unknown parameters:

\begin{itemize}
\item $\beta|y_{1:t},x_{0:t},\phi,\sigma^2_m,\sigma^2_s \sim \mbox{N}(b_n,B_n)$, where
\begin{align*}
b_n &= B_n\left(\frac{1}{\sigma^2_m} \sum^t_{i=1} U_i'\tilde{V}^{-1}(y_i - F_ix_i) + B_0^{-1}b_0\right) \\
B_n &= \left(\frac{1}{\sigma^2_m} \sum^t_{i=1} U_i'\tilde{V}^{-1}U_i + B_0^{-1}\right)^{-1}
\end{align*}
\item $\sigma^2_m|y_{1:t},x_{0:t},\beta,\phi,\sigma^2_s \sim \mbox{IG}(a_{m_n}, b_{m_n})$, where
\begin{align*}
a_{m_n} &= tq/2 + a_{m_0} \\
b_{m_n} &= \frac{1}{2}\sum^t_{i=1} \left((y_i - U_i\beta - F_ix_i)'\tilde{V}^{-1}(y_i - U_i\beta - F_ix_i)\right) + b_{m_0}
\end{align*}
\item $p(\phi|y_{1:t},x_{0:t},\beta,\sigma^2_m,\sigma^2_s) = \Psi(\phi)\mbox{N}_{S_\phi}(\phi; \phi_n,\Phi_n)$, where
\begin{align*}
\phi_n &= \Phi_n\left(\frac{1}{\sigma^2_s} \sum^t_{i=1} \tilde{X}_i'\tilde{W}^{-1}x_i + \Phi_0^{-1}\phi_0\right) \\
\Phi_n &= \left(\frac{1}{\sigma^2_s} \sum^t_{i=1} \tilde{X}_i'\tilde{V}^{-1}\tilde{X}_i + \Phi_0^{-1}\right)^{-1} \\
\tilde{X}_i &= \left(\begin{array}{ccccc}
x_{i-1,1} & x_{i-2,1} & \cdots & \cdots & x_{i-p,1} \\
0 & x_{i-1,1} & \cdots & \cdots & x_{i-(p-1),1} \\
\vdots & 0 & \ddots & & \vdots \\
\vdots & \vdots & \ddots & \ddots & \vdots \\
0 & 0 & \cdots & 0 & x_{t-1,1}
\end{array}
\right) \\
\Psi(\phi) &= \left|\tilde{C_0}\right|^{-1/2}\exp\left(-\frac{1}{2\sigma^2_s}x_0'\tilde{C_0}x_0\right)
\end{align*}
\noindent and $\mbox{N}_{\Omega}(.;\mu,\Sigma)$ represents the pdf of the normal distribution truncated onto the set $\Omega$, with mean and covariance of its corresponding untruncated normal distribution given by $\mu$ and $\Sigma$, respectively.
\item $\sigma^2_s|y_{1:t},x_{0:t},\beta,\phi,\sigma^2_m \sim \mbox{IG}(a_{s_n}, b_{s_n})$, where
\begin{align*}
a_{s_n} &= (p/2)(t+1) + a_{s_0} \\
b_{s_n} &= \frac{1}{2}\sum^t_{i=1} \left((x_i - Gx_{i-1})'\tilde{W}^{-1}(x_i - Gx_{i-1})\right) + \frac{1}{2}x_0'\tilde{C_0}x_0 + b_{s_0}
\end{align*}
\end{itemize}

\clearpage

\bibliographystyle{plainnat}
\bibliography{jarad}

\end{document} 