\documentclass{article}

\usepackage{amsmath}
\usepackage{natbib}
\usepackage{enumitem}
\usepackage{graphicx}

\begin{document}

Presented here is the particle filter known as the \emph{resample-move} algorithm, as described by \citet{Gilk:Berz:foll:2001} and \citet{Berz:gilk:2001}, applied to a dynamic linear model (DLM) \citep{petris2009dynamic}.

\section{Model} \label{sec:model}

We can represent a DLM as a state space model with observation and state equations given by

\begin{align}
y_t &= Fx_t + v_t \label{eqn:obs} \\
x_t &= Gx_{t-1} + w_t \label{eqn:state}
\end{align}

\noindent where $v_t$ and $w_t$ are iid Gaussian random variables with mean 0 and covariance matrices $V$ and $W$, respectively. We also assume $v_t \perp w_{t'}$ for all $t$ and $t'$. $y_t$ is the observed data at time $t$, and $x_t$ is the unknown state at time $t$. We regard $F$ and $G$ as known, and consider the special case where $V = \sigma^2V'$ and $W = \sigma^2W'$, where $V'$ and $W'$ are known and $\sigma^2$ is an unknown parameter. A fully Bayesian model is specified by defining the prior $p(x_0,\sigma^2)$, which we give as \[p(\sigma^2) \sim \mbox{IG}(a_0,b_0) \qquad p(x_0|\sigma^2) \sim \mbox{N}(0,\sigma^2C_0)\]
\noindent where $a_0$, $b_0$, and $C_0$ are known.

\section{Resample-move algorithm} \label{sec:rm}

Let $y_{1:t} = (y_1,y_2,\ldots,y_t)'$ represent the vector of data observed up until time $t$ and let $\theta_t$ be a vector of unknown states and/or parameters at time $t$. We allow $\theta_t$ to change with time, and we consider in particular the case where the dimension of $\theta_t$ changes with time, e.g. if $\theta_t$ contains an evolving trajectory of states. Let $\theta^{+}_t$ be the (possibly empty) set of unknown states entering the model at time $t+1$. Then, we can define $\theta_{t+1} = (\theta_t,\theta^{+}_t)'$. For the DLM described in Section \ref{sec:model}, we might wish to keep track of the entire trajectory of unknown states up to time $t$, in which case $\theta_t = (x_0,x_1,\ldots,x_t,\sigma^2)'$ and $\theta^{+}_t = x_{t+1}$.

We use particle filtering to approximate the posterior distribution $p(\theta_t|y_{1:t})$ through a weighted sample approximation, each sample being termed a \emph{particle}. Let $\theta_{j,t}$ be the vector of unknown states and/or parameters for particle $j$ at time $t$ and $w_{j,t}$ be the weight of particle $j$ at time $t$. We require that $\sum_{j=1}^J w_{j,t} = 1$, where $J$ is the total number of particles.

We initialize the algorithm by drawing $\theta_{j,0}$ from some prior distribution $p(\theta_0)$ and setting $w_{j,0} = 1 / J$ for $j = 1,\ldots,J$. Given a particle sample approximation of the posterior $p(\theta_{t-1}|y_{1:t-1})$ - i.e. $\left(\theta_{j,t-1}, w_{j,t-1}\right)$ for $j=1,\ldots,J$ - we move to a particle sample approximation of $p(\theta_t|y_{1:t})$ by the following steps:

\begin{enumerate}
\item Augmentation and update weights: For each $j = 1,\ldots,J$, draw $\theta^{+}_{j,t-1}$ from some state evolution distribution $p(\theta^{+}_{t-1}|\theta_{j,t-1})$ and define a new particle $\theta^{*}_{j,t} = (\theta_{j,t-1},\theta^{+}_{j,t-1})'$. Then, calculate the unnormalized weight $w^{**}_{j,t} = w_{j,t-1}p(y_t|\theta^{*}_{j,t-1})$.
\item Renormalize the $w^{**}_{j,t}$'s by setting $w^{*}_{j,t} = w^{**}_{j,t} / \sum_{l=1}^J w^{**}_{l,t}$.
\item Resample (optional): For each $j$, sample an index $k_j$ from $\{1,2,\ldots,J\}$ with associated probabilities $\{w^{*}_{1,t},w^{*}_{2,t},\ldots,w^{*}_{j,t},\ldots,w^{*}_{J,t}\}$.
\item {\tt if(}resampling was performed{\tt )\{} \\
\begin{enumerate}[label=\alph*.]
\item Update weights: Set $w_{j,t} = 1 / J$ for all $j$.
\item Move particles: For each $j = 1,2,\ldots,J$, draw $\theta_{j,t}$ from some transition kernel $q(\theta_t|\theta^{*}_{k_j,t})$ with invariant distribution $p(\theta_t|y_{1:t})$.
\end{enumerate}
{\tt \} else \{} Set $w_{j,t} = w^{*}_{j,t}$ and $\theta_{j,t} = \theta^{*}_{j,t}$ for all $j$.\}
\end{enumerate}

The `Resample' step is optional in the sense that if particle weights are not out of balance, it is not necessary to resample the particles because the weighted sample of particles would still provide a good approximation of the posterior. In this case, it would also not be necessary to move the particles, since the problem of degeneracy in the fixed parameters caused by resampling would not be present. By only resampling and moving particles at iterations of the particle filter for which it is necessary, we increase the efficiency of the algorithm. Measures of nonuniformity of particle weights such as effective sample size, coefficient of variation, and entropy can be used to determine when resampling should be performed \citep{Liu:Chen:Wong:reje:1998}. %We use an effective sample size threshold of 80\% of the total number of particles to determine when resample and move particles.

In addition, there are different ways to implement the `Resample' step that may be more efficient under certain conditions. The algorithm as written above implicity uses a \emph{multinomial} resampling scheme, i.e. when we say `sample an index $k_j$ ...'. Additional techniques for resampling exist including \emph{residual}, \emph{stratified}, and \emph{systematic} resampling. \citet{Douc:Capp:Moul:comp:2005} provides an overview of these methods. %We use stratified resampling.

\section{Model Comparison}

Using the algorithm described in Section \ref{sec:rm} we obtain an approximation to $p(\theta_t|y_{1:t})$ through $\sum_{j=1}^J w_{j,t}\delta_{\theta_{j,t}}(x)$, where $\delta_{x_0}(x)$ is the Dirac delta mass function centered at $x_0$. In addition we can use the weighted particle sample to obtain an approximation to the marginal likelihood $p(y_{1:t})$. We first note that given $p(y_{1:t-1})$, $p(y_{1:t})$ can be updated recursively by $p(y_{1:t}) = p(y_t|y_{1:t-1})p(y_{1:t-1})$. Next, we note that an approximation of $p(y_t|y_{1:t-1})$ can be obtained from the weighted particle sample by
\begin{equation}
p(y_t|y_{1:t-1}) \approx \sum_{j=1}^J w_{j,t-1}p(y_t|\theta_{j,t}) \label{eqn:condmarg}
\end{equation}

Having now prescribed a method for approximating $p(y_{1:t})$ sequentially within the resample-move algorithm, we can compare a set of possible models ${M_1,M_2,\ldots,M_p}$ by calculating posterior model probabilities by
\begin{equation}
p(M_i|y_{1:t}) = \frac{p(y_{1:t}|M_i)p(M_i)}{\sum_{i=1}^p p(y_{1:t}|M_i)p(M_i)} \label{eqn:modelcomp}
\end{equation}

\subsection{Example}

Consider three models of the form described in \eqref{eqn:obs} and \eqref{eqn:state}. For all models, we let known values $F = G = V = 1$, and we let $W$ be known at values 0.5, 1, and 1.5 for models $M_1$, $M_2$, and $M_3$, respectively. Data were simulated from $M_2$ with common observation and state equation variance $\sigma^2$ set equal to 1. From this simulated data, true posterior model probabilities can be calculated exactly using equations \eqref{modelcomp} since the predictive distribution $p(y_t|y_{1:t-1})$ is known in this case to follow a Student-t distribution (we let prior model probabilities be equal to 1/3 apiece). We can compare these values with those obtained from running the resample-move particle filter and then using equations \eqref{eqn:condmarg} and \eqref{eqn:modelcomp} to approximate the posterior model probabilities.

100 data sets were simulated from $M_2$ with $\sigma^2 = 1$. For each data set, posterior model probabilities were calculated using both the true predictive distributions (Student-t) and the resample-move particle filter with 100 particles. Figure \ref{fig:probhist} displays histograms of these probabilities and tables \ref{tab:true.propbestmod} and \ref{tab:pf.propbestmod} display the proportion of the simulations for which each model had the highest posterior model probability. For 42\% of the simulations, the model that generated the highest posterior probability using particle filtering was the same as when calculating posterior model probabilities directly (95\% CI: (0.32, 0.52)).

\clearpage

\begin{figure}
\includegraphics[width=1.0\textwidth]{../graphs/mod-post}
\caption{Histograms of posterior probabilities for each model (columns) calculated directly (top row) and using particle filtering (bottom row) for 100 data sets simulated from $M_2$ with $\sigma^2 = 1$.} \label{fig:probhist}
\end{figure}

\begin{table}
\begin{center}
\caption{Proportion of times out of 100 simulations each model generated the highest posterior model probability using direct calculation. Credible intervals were calculated by normal approximation to binomial distribution.} \label{tab:true.propbestmod}
\begin{tabular}{|c|c|c|c|}
\hline
 Model & \% Highest Prob. & lower 95\% CI bound & upper 95\% CI bound \\
 \hline
 $M_1$ & 0.20 & 0.12 & 0.28 \\
 $M_2$ & 0.51 & 0.41 & 0.61 \\
 $M_3$ & 0.29 & 0.20 & 0.38 \\
 \hline
\end{tabular}
\end{center}
\end{table}

\begin{table}
\begin{center}
\caption{Proportion of times out of 100 simulations each model generated the highest posterior model probability using particle filtering. Credible intervals were calculated by normal approximation to binomial distribution.} \label{tab:pf.propbestmod}
\begin{tabular}{|c|c|c|c|}
\hline
 Model & \% Highest Prob. & lower 95\% CI bound & upper 95\% CI bound \\
 \hline
 $M_1$ & 0.22 & 0.14 & 0.30 \\
 $M_2$ & 0.52 & 0.42 & 0.62 \\
 $M_3$ & 0.26 & 0.17 & 0.35 \\
 \hline
\end{tabular}
\end{center}
\end{table}

\clearpage

\bibliographystyle{plainnat}
\bibliography{jarad}

\end{document} 